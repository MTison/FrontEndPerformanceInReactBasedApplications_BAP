%%=============================================================================
%% Conclusie
%%=============================================================================

\chapter{Conclusie}
\label{ch:conclusie}

% TODO: Trek een duidelijke conclusie, in de vorm van een antwoord op de
% onderzoeksvra(a)g(en). Wat was jouw bijdrage aan het onderzoeksdomein en
% hoe biedt dit meerwaarde aan het vakgebied/doelgroep? 
% Reflecteer kritisch over het resultaat. In Engelse teksten wordt deze sectie
% ``Discussion'' genoemd. Had je deze uitkomst verwacht? Zijn er zaken die nog
% niet duidelijk zijn?
% Heeft het onderzoek geleid tot nieuwe vragen die uitnodigen tot verder 
% onderzoek?

Uit het onderzoek dat gevoerd is kunnen we concluderen dat er wel degelijk vele technieken bestaan voor het verbeteren van de frontend performantie. Er zijn oplossingen die niet van het framework zelf komen, maar van een derde partij zoals bepaalde libraries.\\
Het is een goede zaak om een overzicht te hebben van optimalisatie technieken voor een specifiek framework, hier in dit geval React. Waar het onderzoek wat vast loopt is het uitdagen van de reeds bestaande technieken. Alle mogelijke optimalisaties zijn reeds uitgevoerd en onderzocht en om een onderzoek te voeren die werkelijk performantie gaat optimaliseren op een innovatieve manier is een proces van maanden. \\
Deze scriptie is wel een leidraad voor developers die beginnen met React voor hun frontend uitwerkingen. Er kunnen lessen getrokken worden uit het gevoerde onderzoek en zorgen voor een optimaal gebruik van het framework vanaf het begin. 

