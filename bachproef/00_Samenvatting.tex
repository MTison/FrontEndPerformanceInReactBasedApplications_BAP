%%=============================================================================
%% Samenvatting
%%=============================================================================

% TODO: De "abstract" of samenvatting is een kernachtige (~ 1 blz. voor een
% thesis) synthese van het document.
%
% Deze aspecten moeten zeker aan bod komen:
% - Context: waarom is dit werk belangrijk?
% - Nood: waarom moest dit onderzocht worden?
% - Taak: wat heb je precies gedaan?
% - Object: wat staat in dit document geschreven?
% - Resultaat: wat was het resultaat?
% - Conclusie: wat is/zijn de belangrijkste conclusie(s)?
% - Perspectief: blijven er nog vragen open die in de toekomst nog kunnen
%    onderzocht worden? Wat is een mogelijk vervolg voor jouw onderzoek?
%
% LET OP! Een samenvatting is GEEN voorwoord!

%%---------- Nederlandse samenvatting -----------------------------------------

\IfLanguageName{english}{%
\selectlanguage{dutch}
\chapter*{Samenvatting}
\lipsum[1-4]
\selectlanguage{english}
}{}

%%---------- Samenvatting -----------------------------------------------------
% De samenvatting in de hoofdtaal van het document

\chapter*{\IfLanguageName{dutch}{Samenvatting}{Abstract}}

React was sinds enkele jaren nog een nieuwkomer op het toneel voor frontend JavaScript development. Sindsdien is de populariteit van de flexibele library enorm toegenomen en is het één van de meest prominente frontend frameworks geworden. In deze scriptie wordt een onderzoek gedaan naar de aspecten binnen het React landschap dat performantie kan bevorderen op frontend niveau.\\
In de scriptie wordt React uitvoerig ontleed en de meest courante technieken voor het verbeteren van de performantie in het framework worden onderzocht, met bijhorende studie.\\
Dit onderzoek draagt bij tot het in kaart brengen van de belangrijkste factor voor het maken van moderne webapplicaties. De impact die websites op een bedrijf kunnen hebben en een betere kijk op wat het precies allemaal inhoud. Voor deze scriptie is een uitgebreid onderzoek gedaan naar die belangrijke factor, performantie. Hierbij is de scriptie opgedeeld in drie delen. In deel één wordt het framework onder de loop genomen om een beter begrip te krijgen van de structuur en het gebruik van React. Het tweede deel omvat de theorie achter performantie. Daartegenover staat het derde deel van de scriptie die ingaat op de praktische kan van performantie en wat er aan kan gedaan worden om het te optimaliseren als zijde developer.\\
Het resultaat van de scriptie is veelbelovend. Aangezien het internet zo snel evolueert, gaan ook de verwachtingen van  mensen de hoogte in. Er is aangetoond dat met de juiste technieken en voldoende kennis over het framework optimalisaties kunnen toegepast worden.\\
Omtrent verdere onderzoeken zie ik dit in de toekomst nog meer gedaan worden. Het blijft veranderen en de standaarden van nu zullen binnen de kortste tijd weer veranderen en dan is er altijd plaats voor een volgend soortgelijk onderzoek. Daarom niet speciaal voor het React, maar voor de nieuwe opkomende frameworks zoals Preact, vue.js, Flutter,\dots
